\newpage
\section{Galois Groups of Polynomials}
\begin{proposition}
    Suppose $f(x)\in F[x]$ is a monic polynomial with no repeated roots, $K$ is the splitting field of $F[x]$, $f(x)=\prod^m_1(x-\alpha_i)$. Then
    \par (1) $\mathrm{Gal}(K/F)$ is isomorphic to a subgroup $G$ of $S_{\alpha_1,\cdots,\alpha_m}$.
    \par (2) $f(x)\in F[x]$ is irreducible iff $G$ is transitive.
\end{proposition}
\begin{proof}
    \par (1) Let $X=\{\alpha_1,\cdots,\alpha_m\}$. Suppose $\sigma\in\mathrm{Gal}(K/F)$, then $f(\sigma(\alpha_i))=0$, therefore $\sigma(X)\subset X$. Obviously $\sigma(\alpha_i)\neq\sigma(\alpha_j)$ when $i\neq j$. Thus $\sigma|_X\in S_X$. Since $K=F(\alpha_1,\cdots,\alpha_m)$, $\sigma=\tau\in G$ iff $\sigma(\alpha_i)=\tau(\alpha_i)$ for each $i$. Thus $G$ is a subgroup of $S_X$.
    \par (2) Suppose $G$ is transitive on $X$, then $\forany\alpha_i,\alpha_j\in X$, $\exist\sigma\in G$, such that $\sigma(\alpha_i)=\alpha_j$. Therefore $\mathrm{Irr}(\alpha_i,F)=\mathrm{Irr}(\alpha_j,F)$. Therefore in $K[x]$, $f(x)=\prod^m_1(x-\alpha_i)|\mathrm{Irr}(\alpha_i,F)$. Therefore $f(x)=\mathrm{Irr}(\alpha_i,F)$ is irreducible.
    \par Conversely, if $f(x)$ is irreducible, then $\mathrm{Irr}(\alpha_i,F)=\mathrm{Irr}(\alpha_j,F)$ for each $i,j$. Then by proposition \ref{number-of-extensions}, there is an extension $\sigma'$ such that $\sigma'(\alpha_i)=\alpha_j$.
\end{proof}
\begin{definition}[Galois Groups of Polynomials]
    Given $f(x)\in F[x]$, denote its splitting field $K$. The galois group of this polynomial $G(f,F):=\mathrm{Gal}(K/F)$.
\end{definition}
\begin{proposition}
    Suppose $x_1,\cdots,x_n$ transcendental elements, $p_1,\cdots,p_n$ elementary symmetric polynomials of $x_1,\cdots,x_n$, $g(x)=\prod^n_{i=1}(x-x_i)\in F(p_1,\cdots,p_n)[x]$. Then $G(g,F(p_1,\cdots,p_n))\cong S_n$.
\end{proposition}
\begin{proof}
    Let $G=\mathrm{Gal}(F(x_1,\cdots,x_n)/F)$. Then $\forany\sigma\in S_n$, $\sigma$ gives an automorphism on $F[x_1,\cdots,x_n]$ with $\sigma|_F=\mathrm{id}$. Extend it to $F(x_1,\cdots,x_n)$ with 
    $$
    \sigma(p/q)=\sigma(p)/\sigma(q),\quad p,q\in F[x_1,\cdots,x_n]
    $$
    therefore $\sigma\in G$, $S_n\le G$. Since $\sigma(p_i)=p_i$, $F(p_1,\cdots,p_n)\subset\mathrm{Inv}S_n$.
    By proposition \ref{1st-inequality}, proposition \ref{number-of-extensions} and the fact that $F(x_1,\cdots,x_n)$ is the splitting field of $g(x)=\prod_i(x-x_i)\in F(p_1,\cdots,p_n)$, 
    $$[F(x_1,\cdots,x_n):\mathrm{Inv}S_n]=|S_n|=n!\le[F(x_1,\cdots,x_n),F(p_1,\cdots,p_n)]\le n!$$
    therefore $\mathrm{Inv}S_n=F(p_1,\cdots,p_n)$.
\end{proof}
\begin{remark}
    From now on, assume $F$ is of characteristic 0.
\end{remark}
\begin{definition}[Roots of Unity]
    The roots of $x^n=1$ are called n-th root of unity. If a n-th root of unity $\theta$ satisfies
    $$
    \theta^n=1,\quad\theta^m\neq 1,\quad \forany m, 0<m<n
    $$
    then it is called a n-th primitive root of unity.
\end{definition}
\begin{proposition}
    Suppose $\theta$ a n-th primitive root of unity. Then $\theta^k$ is primitive iff $(k,n)=1$.
\end{proposition}
\begin{proof}
    Suppose $K$ the splititng field of $x^n-1\in F[x]$. Since $(nx^{n-1},x^n-1)=1$, $x^n-1$ is separable, therefore has no repeated roots. Therefore the set of roots is $\{1,\theta,\cdots,\theta^{n-1}\}=\left<\theta\right>$. The rest is clear since it is a cyclic group.
\end{proof}
\begin{definition}[Euler Function]
    $\varphi(n)$ denotes the number of coprimes that are less than $n$.
\end{definition}
\begin{remark}
    It is easy to see that $\varphi(n)$ also denotes the number of n-th primitive roots of unity.
\end{remark}
\begin{definition}[Cyclotomic Polynomial]
    n-th cyclotomic polynomial is defined as
    $$
    \phi_n(x)=\prod^{\varphi(n)}_{i=1}(x-\theta_i)
    $$
\end{definition}
\begin{remark}
    The following proof is omitted since lack of time.
\end{remark}
\begin{proposition}
    $\forany n\in\mathbb{N}$,
    \par (1) $x^n-1=\prod_{d|n}\phi_d(x)$
    \par (2) $\phi_n(x)$ is irreducible on $\mathbb{Q}[x]$.
\end{proposition}
\begin{proposition}\label{tt1}
    $G(\phi_n,\mathbb{Q})$ is abelian.
\end{proposition}
\begin{proposition}\label{tt2}
    Suppose $a\in\mathbb{Q}$, $G(x^n-a,\mathbb{Q})$ is abelian.
\end{proposition}
\begin{proposition}\label{tt3}
    If $F$ contains n-th roots of unity, and $K$ a Galois extension such that $\mathrm{Gal}(K/F)$ is a cyclic group of degree $n$, then $\exist\epsilon\in K$ such that $K=F(\epsilon)$ and $\epsilon^n=a\in F$.
\end{proposition}